\chapter{Valorisation du travail réalisé}

Ici, nous allons mettre en valeur notre code en explicitant les différents algortihmes utilisés et notre réflexion 
ayant mené à ce projet final. Pour cela, nous allons notamment voir l'exemple d'un bug que l'on a eu, la façon 
dont on l'a résolu, ainsi que les optimisations ayant permis d'éviter au maximum les redondances dans le code.

\section{Prise en main sur une seule machine}
La première étape du projet a consisté à créer un pong basique sans le réseau. Nous ne sommes
pas parti de rien, le sujet étant 
donné avec du code. Nous avons pu apprendre à nous servir de swing,
l'API de java pour faire une interface,
et commencer la première partie qui a consisté à créer un pong multijoueur, jouable sur une seule machine.
Notre première décision fut de changer l'aspect graphique du jeu, se rapprochant plus du pong originel.
\\
Nous avons repris l'implémentation donnée dans le code de base et le sujet, une 
classe pong qui gère tout le jeu, 
une classe Window qui affiche le rendu et par laquelle l'appel au réseau, contenu dans 
une classe à part, se fera plus tard. Une fonction Main 
lance le jeu et décide ce que feront
le serveur et le client. Enfin, il y a deux classes, Racket et Ball, héritant toutes deux de PongItem, interface 
contenant 3 paramètres 
communs aux 2 objets, ``width'', ``height'' et ``position''.
\\
Le premier problème fut de gérer la collision et le rebond de la balle sur les raquettes. 
Pour cela, nous avons utilisé un algorithme prenant le problème de revers, 
c'est-à-dire que nous avons cherché à savoir si la balle n'était pas sur la 
raquette. En pensant différement, nous avons donc pu contourner le problème.
Le second, le problème lié au rebond, était du à la diférenciation de l'axe x ou l'axe y sur 
lesquels la balle devait rebondir. Avec notre nouveau mode de pensée utilisé et appliqué sur 
le rebond, nous avons pu résoudre ce problème aussi.
Plus tard, dans le projet, nous avons de nouveau modifié le rebond pour avoir quelque chose 
de plus complexe, nous avons coupé la 
raquette en cinq et, selon la partie sur laquelle la balle rebondit, l'orientation de ce dernier et la vitesse 
sera différente.


\section{Passage en réseau}
Une fois le mode de jeu multijoueur mis en place de façon locale, il aura fallu mettre en place le réseau. 
Pour cela, comme expliqué précédement, une classe Réseau a été créée, contenant 2 fonctions, send et read. 
L'envoi et la réception de données et le système d'ack ayant déjà été détaillé, nous allons ici nous concentrer sur le 
mode d'envoi de données choisi, la raison de la multiplicité des fonctions read et send ainsi que celle de l'unicité 
de la fonction ack.
\\
Après la rédaction de plusieurs fonctions send et read, afin d'envoyer et recevoir les données de la racket, de la balle 
et du score max à établir, nous avons réussi à factoriser le tout en obtenant 4 fonctions. Ainsi, les fonctions ``read'' et 
``send'' sont là pour envoyer des coordonnées. Il peut s'agir des coordonnées de position de la racket, de position de la balle 
ou de vitesse de celle-ci. Voici les prototypes de ces 2 fonctions:
\begin{itemize}
 \item \textit{public void send(Socket sock, Point point, String coor)}
 \item \textit{public void read(Socket sock)}
\end{itemize}

Les 2 fonctions prennent ainsi nécessairement la socket pour l'envoi/réception des données. La fonction ``send'', quant à 
elle, prend la variable ``point`` ainsi que le String ''coor`` déterminant l'origine du point ainsi que la 
coordonnée de celui-ci à envoyer.
\\
Les 2 autres fonctions send/read sont les suivantes:
\begin{itemize}
 \item public void sendMaxScore(Socket sock, int scoreMax)
 \item public void readMaxScore(Socket sock)
\end{itemize}

Celles-ci sont appelées une seule fois au début de la partie, contrairement aux 2 précédentes, mais fonctionnent sur le même 
principe, seul le type de donnée diffère. Ici, il s'agit d'un int. Une fonction readMaxScore spécifique a donc été créée afin 
de réceptionner la valeur correctement.
\\
La fonction ack, intégrée dans un do while, comme expliqué précédemment, est générique et s'applique à absoluement tout type 
de donnée, à condition qu'il respecte ces différentes conditions. Tout d'abord, lors du read, la dernière étape 
doit être l'envoi d'une chaîne de caractère unique servant à identifier le type de donnée qui a été réceptionné. Ensuite, 
cette chaîne de caractère doit être passée en paramètre à la fonction ''ack`` qui la comparera avec celle reçue et renverra 
''True`` si se sont les mêmes, ''False`` sinon, entraînant l'envoi de l'information à nouveau. Voici le prototype de 
cette fonction:

\begin{itemize}
 \item public boolean ack(Socket sock, String message)
\end{itemize}

Cette fonction récupère ainsi les informations via la socket et compare la chaîne de caractères obtenue avec la chaîne 
''message`` passée en paramètre.

\section{Extension}

\subsection{Eviter la triche}
Pour éviter la triche, nous avons repris l'algorithme et les explications données dans le sujet. 
On calcule ainsi la position attendue par la balle à l'aide de sa vitesse, puis on 
la compare avec celle reçue. Si ce sont les mêmes, il n'y a pas de triche.
On ajoute l'appel à cet algorithme après les boucles traitant la réception des données.
\\
Plus tard, nous avons ajouté des balles bonus qui augmentent ou diminuent la vitesse. 
Il fallait un signal pour avertir que les raquettes
étaient affectées par les bonus. Grâce à notre implémentation, nous avons un paramètre de type int dans 
la structure racket indiquant si une 
raquette est touchée par un bonus. Il suffit donc de vérifier si cet int était à la bonne valeur pour les deux raquettes.

\subsection{Les balles spéciales et bonus}
La seconde extension que nous avons ajouté fût les bonus. Nous avons commencé par un bonus agrandissant 
la raquette courante du joueur, en prenant en compte le fait qu'il faut gérer une collision différente. 
Il faut donc se servir 
de notre fonction implementée pour les collisions avec en argument
ball et racket. Les bonus étant tous des types de Ball particuliers, ils héritent tous de cette classe. 
\\
Ne voulant pas surcharger le réseau de données, nous avons décidé que chaque jeu gèrera lui-même tous 
les bonus et qu'ils apparaitraient
donc au bout d'un certain nombre de points marqués par les deux joueurs. Ainsi, on garantit 
la synchronisation, le bonus arrivant sur la raquette 
ayant marqué le dernier
point.
\\
Nous avons créé plusieurs données membres pour une classe nomée BallBonusLarger. Tou d'abord, un booléen qui nous 
permet de savoir si le bonus a été créé, qui sert afin de ne pas écrire de données sur des bonus non 
instanciés, grâce à un if.
Puis, un tableau de 2 int, qui correspond au score courant de la partie lors de la création du bonus. Cette donnée 
nous sert à ne pas créer de nouveaux bonus lorsqu'un est déjà instanciée et que le score contenu dans le tableau 
correspond au score où celui-ci a été créé. Cela évite que les bonus apparaissent sans cesse tant que le score n'est pas 
modifié. 
Enfin, un entier qui définit la position Y du bonus, qui correpondra à la hauteur de la dernière balle perdue.
\\
Après cela, nous avons crée de nouveaux bonus. L'un d'entre eux réduit la raquette du joueur qui l'obtient, tandis que 
les autres augmentent sa vitesse ou au contraire, la réduise. 
Pour ce faire, nous avons créé une classe, située au-dessus dans la hiérarchie, nommée 
BallBonus, héritant de Ball. Ainsi, toutes les classes des 
différents bonus héritent de BallBonus. 
Puis, nous avons mis dans cette classe ce que nous avions précédement fait dans 
BallBonusLarger avec un 
constructeur permettant de choisir l'image du bonus. Enfin, nous avons crée des fonctions utilisées lors 
de la création et l'affichage,
utilisant des données membres pouvant être changée par le constructeur, nous permettant de gérer différents paramètres aisément, 
comme le nombre de points à marquer avant que le bonus apparaisse, l'effet que le bonus aura sur la raquette et le sens 
dans lequel il ira. Il se déplacera ainsi sur celui qui a marqué le dernier point s'il s'agit d'un bonus 
considéré comme positif, ou l'adversaire s'il s'agit d'un malus.

\subsection{Menu}
En dernière partie, nous avons décidé d'ajouter un menu qui nous faciliterait, autant à nous, programmeurs, qu'à 
l'utilisateur, 
le lancement d'une partie de pong. Après des recherches pour comprendre la façon dont swing gére les menus, nous 
avons donc créé
une 
classe Menu.
Pour intégrer les données postées par l'utilisateur dans le menu au reste du code, nous 
avons créé des données membres, dans la classe Main, 
accessibles par
toutes les classes permettant de gérer les différentes options du menu. 
Nous avons aussi créé un thread, que nous avons utilisé 
dans le Main, pour attendre que l'utilisateur lance le jeu avant de commencer à lancer le réseau, puisqu'il nous 
fallait les réponses de celui-ci.
Une des options du menu en début de partie est le score max que doit atteindre un joueur pour gagner une partie, et donc il 
fallait aussi gérer les joueurs qui ne mettraient pas la même valeur, alors qu'ils jouent ensemble. Pour contourner 
ce problème, 
nous avons décidé que ce serait le serveur qui choisirait le score max et qui l'enverrait en début de partie au client.
Enfin, nous avons créé une boîte de dialogue pour la fin d'une partie, lancée lorsqu'un joueur atteint le score maximum et 
proposant deux options: une permettant de rejouer, l'autre de fermer la fenêtre du jeu. La seule chose 
à réinitialiser pour une nouvelle partie
fût le score, et pour la fermeture de la fenêtre, on envoie un entier qui nous permet de sortir 
de la boucle while de Window, 
là où toute la partie affichage se fait.