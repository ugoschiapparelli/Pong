\chapter{Introduction}

Ce rapport a pour ambition de présenter notre projet effectué dans le cadre de l'UE Réseau et Projet
de Programmation 2. Il s'agit d'un Pong, l'un des tout premiers jeux vidéos, développé ici en Java
et offrant la possibilité de jouer seul face à une IA ou en réseau, face à un autre joueur. Le Pong est
une simulation de tennis de table. Ainsi, une balle rebondit entre les raquettes de 2 différents
joueurs, situées aux extrémités gauche et droite de l'écran. Ceux-ci se renvoient ainsi mutuellement
la balle jusqu'à ce que l'un d'eux marque un point en la faisant passer derrière la raquette de
l'adversaire. La personne ayant marqué un point voit ainsi son score augmenter de 1. Les raquettes
sont contrôlées à l'aide des touches directionnelles « haut » et « bas » et bougent de façon verticale
sur l'écran.
Nous allons ainsi par la suite présenter les différentes fonctionnalités de notre Pong, celles prévues
originellement, la mise en place d'un pong multijoueur en réseau, mais aussi de nouvelles,
développées par envies personnelles. Les fonctions et différents algorithmes composant ce projet seront ensuite explicités.
Enfin, un diagramme des classes et une conclusion viendront résumer notre projet et
permettront d'effectuer un bilan, traitant les limites et autres fonctionnalités supplémentaires qu'il
serait envisageable d'implémenter.
